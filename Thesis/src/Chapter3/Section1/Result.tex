\subsection{実験結果および考察}
5回の燃焼器か試験を実施した。燃焼気化試験を表に示す。
図にNo.1の燃焼の陽数を示す。着火及び燃焼の持続に成功した。
No.1の燃焼気化中の計測データを時間履歴を図10に示す。
インジェクタ上流の酸素温度は、沸点異常であり、LOXの安定供給はできなかった。
ノズル上流及び下流の温度は、K熱電対の測定可能上限1280℃を超えており、実験後に熱電対を確認した所、焼損していることがわかった。
プリバーナ部のベークライトスペーサ及びベークライトのバッフルプレートは焼損しており、酸素の加熱に寄与したことがわかる。
バッフルプレートはインジェクタ直下が激しく焼損しており、酸素がいき良いよく衝突していることがわかった。
\\
No.5の燃焼気化中の計測データの時間履歴を図に示す。
インジェクタ上流の酸素温度は、沸点以下に下がっており、LOXを安定に供給はできた。
タービン流量計は不具合のため計測できていないが、オリフィスもしくはインジェクタの差圧から流量を見積もった結果から、実験後半では比較的一様にLOXを供給できていることがわかる。
図に実験後のバッフルプレートの様子を示す。
焼損が激しく孔同士が繋がったことがわかる。
\\
燃焼気化試験結果のまとめを表に示す。また、燃料グレインおよびスペーサ、バッフルプレート、グラファイトノズルの質量の変化のまとめを表に示す。

No.1~3の実験では、インジェクタ上流の酸素温度は沸点以上で、安定的にLOXを供給できなかった。
No.4および5では、実験後半で沸点以下に下がっており、LOXを供給することができた。
LOX流量は、基本的には気体混じりの状態が多く、正確には計測できていない。
今回はインジェクタ上下さあつを用いてデータを整理した。
O/Fは、目標の50からは大きく外れている。
\\
グラファイトノズルは全く焼損していないが、ベークライトのスペーサとバッフルプレートは焼損が激しかった。
そのため、大きな流量および長時間での実験が実施できなかった。
\\
No.5no試験では、燃焼の様子の高速度カメラでの撮影に成功した。プリバーナ部分での逆流が発生しており、プリバーナ部で激しく混合していることがわかった。
\\

