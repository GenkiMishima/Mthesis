\subsection{実験結果および考察}
図\ref{fig:S2Case5Grain}にNo.5の燃焼終了後の燃料グレインを示す。
No.1の気化試験の計測データの時間履歴を図\ref{S2Case1}に示す。
流し始めのインジェクタ上流温度が沸点以上となっているが、
インジェクタ上流温度が沸点を下回ると同時に流量が安定している。
これよりインジェクタ上流部では気液が混在した酸素が流入していることが予想される。
ノズル付近の温度が酸素流入直後は大きな値を示しているため、着火が成功していることが確認できる。
その後すぐに沸点付近に下降し沸点と同じ値になっている。
スペーサとバッフルプレートに使用したグラファイトはほとんど焼損がなく、燃焼に寄与していないことが確認できた。
\\
No.5の気化試験の計測データの時間履歴を図\ref{S2Case5}に示す。
本試験もインジェクタ上流温度が沸点を下回ると同時に流量が安定しており、
ノズル付近の温度履歴より、着火が成功していることが確認できる。
No.1~3の$7.5mm$グレインのケースと同様に温度が下降しているが、100℃付近になっている。
この試験でもスペーサとバッフルプレートの焼損がなかった。
\\
燃焼気化試験結果のまとめを表に示す。また、燃料グレインおよびスペーサ、バッフルプレート、グラファイトノズルの質量の変化のまとめを表に示す。

すべての試験で流量が$0.05kg/s$付近で安定して供給できた。
No1~3の試験ではO/Fが140前後と目標値50に対して大きく上回っており、ノズル付近では酸素温度が沸点と同じになっている。
一方No4~8ではO/Fは40程度で目標値と近い値になり、ノズル付近では温度が100℃以上と水の沸点以上になったため、
排気ガスは完全に気化できたことが確認できた。
\\

