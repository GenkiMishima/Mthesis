\subsection{実験結果および考察}
本実験ではすべてのケースで着火が確認できた。
図にNo.5の燃焼終了後の燃料グレインを示す。
No.2の燃焼気化中の計測データの時間履歴を図に示す。
流し始めのインジェクタ上流温度が沸点以上となっているため、流入する酸素状態は気液が混在しているが、インジェクタ上流温度が沸点を下回ると同時に流量が安定している。
ノズル付近の温度が酸素流入直後は大きな値を示しているため、着火が成功していることが確認できる。その後すぐに沸点付近に下降し沸点と同じ値になっている。
スペーサとバッフルプレートに使用したグラファイトはほとんど焼損がなく、燃焼に寄与していないことが確認できた。
\\
No.5の燃焼きかちゅうの計測データの時間履歴を図に示す。
本試験もインジェクタ上流温度が沸点を下回ると同時に流量が安定しており、ノズル付近の温度履歴より、着火が成功していることが確認できる。
No.1~3の$7.5mm$グレインのケースと同様に温度が下降しているが、90℃付近になっている。
この試験でもスペーサとバッフルプレートの焼損がなかった。
\\
燃焼気化試験結果のまとめを表に示す。また、燃料グレインおよびスペーサ、バッフルプレート、グラファイトノズルの質量の変化のまとめを表に示す。

すべての試験で流量が$0.05kg/s$付近で安定して供給できた。
No1~3の試験ではO/Fが140前後と目標値50に対して大きく上回っており、ノズル付近では酸素温度が沸点と同じになっている。
一方No4~8ではO/Fは40程度で目標値と近い値になり、ノズル付近では温度が100℃前後と設計通りの値となった。
\\
グラファイトは全く焼損しておらず、燃焼に寄与していないことが確認できる。

