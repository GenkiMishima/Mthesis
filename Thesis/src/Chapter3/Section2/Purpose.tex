\subsection{実験目的、設定条件}
可視化実験ではスペーサとバッフルプレートに使ったベークライトが焼損していたこと、試験前の冷却が不十分であったため、流量が不安定になったことが問題点になり、長時間の燃焼が行えなかった。
可視化実験ではO/Fが目標値の50を大きく外れていた。
本実験ではスペーサとバッフルプレートにグラファイトを使用し、冷却時間も十分に行った。
またバッフルプレートの孔数を4個,孔径$\phi5.0mm$
本実験ではLOXを安定に供給し、本気化器の基礎データの取得を目的とした実験を行った。
LOX流量が$0.05kg/s$程度になるようにHe圧力を設定し、可視化実験よりO/Fをあげるために、燃料グレインは可視化実験より短い$7.5mm,15mm$の2種類を使用し試験を行った。
プリバーナ部の燃料グレインとスペーサの配置図を図に示す。
本実験は表\ref{TestCondition}の設定条件で行った。
\begin{table}[htb]
\begin{center}
\caption{設定条件}
\small
\begin{tabular}{|c|c|c|c|c|c|} \hline
No. & 設定He圧力 & 燃焼時間 & \multicolumn{3}{|c|}{グレイン(PMMA)} \\ \cline{2-6}
 & MPaA & s & 長さ[mm] & 外径[mm] & 内径[mm]  \\ \hline
1 & 3.4 & 5.0 & & & \\ \cline{1-3}
2 & 4.9 & 8.0 &7.5 & & \\ \cline{1-3}
3 & 3.6 & 8.0 & & & \\ \cline{1-4}
4 & 3.1 & 5.0 & & & \\ \cline{1-4}
5 & 5.0 & 6.5 & & 50 & 30 \\ \cline{1-4}
6 & 4.3 & 6.5 & 15 & &  \\ \cline{1-4}
7 & 4.0 & 6.5 & & &  \\ \cline{1-4}
8 & 3.3 & 6.5 & & &  \\ \hline
\end{tabular}
\label{tab:TestCondition}
\end{center}
\end{table}
