\chapter{プリバーナ方式液体酸素気化実験}
\newcommand{\FigAddThree}{./src/Chapter3/Figure}
\section{可視化実験}
\subsection{実験目的、設定条件}
本実験ではO/F=50以上での酸素とPMMAの反応となるため、燃焼器内部の状態を確認した。
燃焼器外殻が透明な供試体を使用し、高速度カメラによる燃焼室内部の可視化を目的とした実験を行った。
液体酸素流量が$0.02kg/s$程度になるようにHe圧力を設定し、燃料グレインは$15mm,30mm$の2種類を使用し試験を行った。
本実験は表\ref{S1TestCondition}の設定条件で行った。

\subsection{実験結果および考察}
5回の燃焼器か試験を実施した。
No.1の燃焼気化中の計測データを時間履歴を図\ref{fig:S1Case1}に示す。
インジェクタ上流の酸素温度は、沸点異常であり、LOXの安定供給はできなかった。
ノズル上流及び下流の温度は、K熱電対の測定可能上限1280℃を超えており、実験後に熱電対を確認した所、焼損していることがわかった。
プリバーナ部のベークライトスペーサ及びベークライトのバッフルプレートは焼損しており、酸素の加熱に寄与したことがわかる。
バッフルプレートはインジェクタ直下が激しく焼損しており、酸素がいき良いよく衝突していることがわかった。
\\
No.5の燃焼気化中の計測データの時間履歴を図\ref{fig:S1Case5}に示す。
インジェクタ上流の酸素温度は、沸点以下に下がっており、LOXを安定に供給はできた。
タービン流量計は不具合のため計測できていないが、オリフィスもしくはインジェクタの差圧から流量を見積もった結果から、
実験後半では比較的一様にLOXを供給できていることがわかる。
図\ref{fig:S1AfBaffle}に実験後のバッフルプレートの様子を示す。
焼損が激しく孔同士が繋がったことがわかる。
\\
燃焼気化試験結果のまとめを表\ref{tab:S1Result}に示す。
また、燃料グレインおよびスペーサ、バッフルプレート、グラファイトノズルの質量の変化のまとめを表\ref{tab:S1DifSBN}に示す。
No.1~3の実験では、インジェクタ上流の酸素温度は沸点以上で、安定的にLOXを供給できなかった。
No.4および5では、実験後半で沸点以下に下がっており、LOXを供給することができた。
LOX流量はタービン流量計による計測ができていない。
今回はインジェクタ上下流差圧を用いてデータを整理した。
O/Fは、目標の50からは大きく外れている。
\\
グラファイトノズルは全く焼損していないが、ベークライトのスペーサとバッフルプレートは焼損が激しかった。
そのため、大きな流量および長時間での実験が実施できなかった。
\\
No.5試験では、燃焼の様子の高速度カメラでの撮影に成功した。
図\ref{fig:InsideofChamber}にNo.5の燃焼の様子を示す。
白く発光しているところが火炎が存在する領域であるが、
インジェクター直下及びバッフルプレート直上にて激しく発光しているため、
火炎が滞留していることが確認できる。
また実験映像よりプリバーナ部分での逆流が確認でき、
プリバーナ部で激しく混合していることがわかった。

\input{./src/Chapter3/S1Conclusion}


\section{実証機モデルの性能計算}
実験によって得られた計測データを元に実証機モデルの性能計算を評価し、
実証器モデルの寸法を決定した。
性能計算は以下の要領で計算を行った。
\begin{itemize}
\item 酸化剤質量流束より燃料流量を推定
\item 燃焼室圧力、O/FよりnasaCEAで化学平衡計算
\item 計算結果より平均燃料グレイン形状を推定
\item 燃焼終了時間まで繰り返す
\end{itemize}
酸化剤質量流束は燃焼室に噴射する酸化剤質量流量と燃料ポート断面積から求めた。
検討しているハイブリッドロケットエンジンの酸化剤質量流量は最大$10[kg/s]$を想定しているため(\ref{})、
本検討でも酸化剤質量流量は$10[kg/s]$とした。

実証機の設計諸元\ref{tab:RealModel}と寸法及び概略図を表と図に示す。


\begin{table}[htb]
\begin{center}
\caption{可視化実験設定条件}
\small
\begin{tabular}{|c|c|c|c|c|c|} \hline
No. & 設定He圧力 & 燃焼時間 & \multicolumn{3}{|c|}{グレイン(PMMA)} \\ \cline{2-6}
 & MPaA & s & 長さ[mm] & 外径[mm] & 内径[mm]  \\ \hline
1 & 1.2 & 3 & & & \\ \cline{1-3}
2 & 1.2 & 3 & 15 & & \\ \cline{1-3}
3 & 2.0 & 5 & & 50 & 30  \\ \cline{1-4}
4 & 1.2 & 5 &30  &  & \\ \cline{1-3}
5 & 1.2 & 4 &  &  &  \\ \hline
\end{tabular}
\label{tab:S1TestCondition}
\end{center}
\end{table}

\begin{table}[htb]
\begin{center}
\caption{設定条件}
\small
\begin{tabular}{|c|c|c|c|c|c|} \hline
No. & 設定He圧力 & 燃焼時間 & \multicolumn{3}{|c|}{グレイン(PMMA)} \\ \cline{2-6}
 & MPaA & s & 長さ[mm] & 外径[mm] & 内径[mm]  \\ \hline
1 & 3.4 & 5.0 & & & \\ \cline{1-3}
2 & 4.9 & 8.0 &7.5 & & \\ \cline{1-3}
3 & 3.6 & 8.0 & & & \\ \cline{1-4}
4 & 3.1 & 5.0 & & & \\ \cline{1-4}
5 & 5.0 & 6.5 & & 50 & 30 \\ \cline{1-4}
6 & 4.3 & 6.5 & 15 & &  \\ \cline{1-4}
7 & 4.0 & 6.5 & & &  \\ \cline{1-4}
8 & 3.3 & 6.5 & & &  \\ \hline
\end{tabular}
\label{tab:S2TestCondition}
\end{center}
\end{table}

\begin{figure}
\centering
\caption{実験結果}
\includegraphics[width=10cm]{\FigAddThree/png/S1Result.png}
\label{tab:S1Result}
\end{figure}

\begin{figure}
\centering
\caption{各部材の質量変化量}
\includegraphics[width=10cm]{\FigAddThree/png/S1DifSBN.png}
\label{tab:S1DifSBN}
\end{figure}

\begin{figure}
\centering
\includegraphics[width=10cm]{\FigAddThree/png/S1Case1.png}
\caption{実験結果}
\label{fig:S1Case1}
\end{figure}

\begin{figure}
\centering
\includegraphics[width=10cm]{\FigAddThree/png/S1Case5.png}
\caption{実験結果}
\label{fig:S1Case5}
\end{figure}

\begin{figure}
\centering
\includegraphics[width=10cm]{\FigAddThree/png/S1AfBaffle.png}
\caption{燃焼試験終了後のバッフルプレート}
\label{fig:S1AfBaffle}
\end{figure}

\begin{figure}
\centering
\includegraphics[width=10cm]{\FigAddThree/png/InsideofChamber.png}
\caption{燃焼室内部の様子及び燃料流の挙動}
\label{fig:InsideofChamber}
\end{figure}


\begin{figure}
\centering
\includegraphics[width=8cm]{\FigAddThree/png/S2Case1.png}
\caption{実験結果}
\label{fig:S2Case1}
\end{figure}

\begin{figure}
\centering
\includegraphics[width=8cm]{\FigAddThree/png/S2Case5.png}
\caption{実験結果}
\label{fig:S1Case5}
\end{figure}

\begin{figure}
\centering
\includegraphics[width=10cm]{\FigAddThree/png/S2Case5Grain.png}
\caption{燃焼試験終了後の$15mm$燃料グレイン}
\label{fig:S2Case5Grain}
\end{figure}
