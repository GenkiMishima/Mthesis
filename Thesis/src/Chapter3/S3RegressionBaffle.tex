\subsection{燃料形状と後退速度}
燃料グレインの後退速度計測には燃焼試験前後の重量と
グレイン半径方向の厚さより求める2種類の方法で計測し、
この計測方法はそれぞれ平均値と局所値の計測に対応している。
図\ref{fig:RegressionRate}は本実験の固体燃料後退速度-酸化剤質量流束の関係である。
軸流型GOX/PMMAのバッフルプレートがついていない燃料後退速度実験値と比べて、
本実験の全てにおいて高い燃料後退速度を計測している。
燃焼試験終了後の燃料グレインを図\ref{AfGrain75mm}と図\ref{AfGrain15mm}に示す。
$7.5mm$のケースでは燃え残りグレイン形状は燃料表面の中心部分が主に減少している。
これは燃料グレインの上流部と下流部にグラファイトスペーサを配置しており、
グレイン表面の流れ場がキャビティ流れのようになっていたと考えられる。
さらにこのケースにおいては参考にした、バッフルプレートがない軸流型GOX/PMMAの燃料後退速度とよく一致している。
一方$15mm$のケースでは燃焼終了後の燃料グレインからは
上流部に放射状に削れる領域と下流部にえぐるように削れる領域の2つが観測できる。
これより燃焼室内部では上流部と下流部に2つの異なる流れ場が存在したと考えられる。
加えて、これら2つを分ける点は燃料グレイン上端から$7mm$の付近に明確に確認することができる。
軸流型GOX/PMMAのバッフルプレートがついていない燃料後退速度実験値と比べて、
この燃料グレインの平均燃料後退速度は大きな値を示した。
一方、上流部の局所燃料後退速度は参考にした燃料後退速度とよく一致しており、
下流部の局所燃料後退速度が非常に高い値を示している。
可視化実験の映像及びこの燃料グレイン残渣より、
本実験ではバッフルプレートの影響が支配的であったことが確認できる。
