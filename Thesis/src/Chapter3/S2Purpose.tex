\subsection{実験目的、設定条件}
可視化実験ではスペーサとバッフルプレートに使ったベークライトが焼損していたこと、試験前の冷却が不十分であったため、流量が不安定になったことが問題点になり、長時間の燃焼が行えなかった。
可視化実験ではO/Fが目標値の50を大きく外れていた。
本実験ではスペーサとバッフルプレートにグラファイトを使用し、冷却時間も十分に行った。
またバッフルプレートの孔数を4個,孔径$\phi5.0mm$
本実験ではLOXを安定に供給し、本気化器の基礎データの取得を目的とした実験を行った。
LOX流量が$0.05kg/s$程度になるようにHe圧力を設定し、可視化実験よりO/Fをあげるために、燃料グレインは可視化実験より短い$7.5mm,15mm$の2種類を使用し試験を行った。
プリバーナ部の燃料グレインとスペーサの配置図を図に示す。
本実験は表\ref{tab:S2TestCondition}の設定条件で行った。
