\section*{\large 問題1}
\ 
配布資料より、各パラメータは以下のようになる。
\begin{table}[htb]
 \begin{center}
  \caption{各仮定値及びJR100のパラメータ}
  \begin{tabular}{|l|r|} \hline
   \multicolumn{2}{|c|}{|JR100|} \\ \hline
   推力比:$sigma$ (=$\frac{V_n}{V_j}$,$V_n=0$)[-] & 0 \\ \hline
   低圧タービン断熱効率:$\eta_t$ [-] & 0.88 \\ \hline
   ファン断熱効率:$eta_f[-]$ & 0.852 \\ \hline
   バイパス効率:$eta$[-] & 0.74976 \\ \hline
   圧縮機での定圧比熱:$c_{pc} [j/kg K]$ & 1004 \\ \hline
   タービン及びノズルでの定圧比熱:$c_{pt} [J/kg K]$ & 1155 \\ \hline
   圧縮機入口全温:$T_t[K]$ & 288.2 \\ \hline
   タービン出口全温:$T_{t4}[K]$ & 983.2 \\ \hline
   排気静温:$T_{j}[K]$ & 840 \\ \hline
   圧縮機比熱比:$\kappa_c [-]$ & 1.4 \\ \hline
   $\kappa_c/{\kappa_c -1}$ & 3.5 \\ \hline
  \end{tabular}
 \end{center}
\end{table}
また、最適エネルギー分配率$\lambda_{op}$及び、最適推力比$\tau_{op}$を求める式は、
\begin{equation}
 \lambda_{op}=\frac{\mu(\eta^2-\sigma^2)}{\eta(1+\mu \eta)}
\end{equation}
\begin{equation}
 \tau_{op}=\frac{\sqrt{1-\lambda_{op}} -\sigma +(\sqrt{\mu(\mu\sigma^2+\eta\lambda_{op})}-\mu\sigma)}{1-\sigma}
\end{equation}
である。これより得た最適エネルギー分配率$\lambda_{op}$及び最適推力比$\tau_{op}$の値を以下に示す。
\begin{table}[htb]
 \begin{center}
  \caption{各バイパス比にエネルギーおける最適エネルギー分配率及び最適推力比}
  \begin{tabular}{|r|r|r|} \hline
  バイパス比$\mu$ [-] & 最適エネルギー分配率$\lambda_{op} [-] $ & 最適推力比$\tau_{op}[-]$   \\ \hline
  0  & 0            &  1            \\ \hline
  1  & 0.42849305   &  1.322784941  \\ \hline
  2  & 0.599923185  &  1.580987033  \\ \hline
  6  & 0.818134202  &  2.344900851  \\ \hline
  10 & 0.882319714  &  2.915064322  \\ \hline
  15 & 0.91834335   &  3.499485676  \\ \hline
  20 & 0.937481244  &  3.999399955  \\ \hline
  \end{tabular}
 \end{center}
\end{table}
\\
\begin{figure}[t]
 \begin{center}
  \includegraphics[width=10.0cm]{eps/body1_1.eps}
  \caption{バイパス比との関係}
 \end{center}
\end{figure}

%\begin{equation}
%\alpha M = F
%\end{equation}
%\\
%\begin{figure}[b]
% \begin{center}
%  \includegraphics[width=7.0cm]{png/apple.eps}
%  \caption{Title}
%  \label{label}
% \end{center}
%\end{figure}
\\
