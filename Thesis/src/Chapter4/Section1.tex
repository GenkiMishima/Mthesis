\section{実証機モデル設計計算及び重量計算}
実験によって得られた計測データを元に実証機モデルの設計計算を行った。
本計算においては燃料グレインの局所的な後退及び燃焼効率は考慮しておらず、
燃料グレインの後退速度はどの計測点においても等しいと仮定し、
燃焼効率は100\%になると仮定した。
ロケットの燃焼機は慣例的に薄肉円筒圧力容器として扱うことが多く、
本設計計算においてもこの慣例に従う。
軸方向の内圧による力は$\pi r^2 P$となり、これに釣り合うように軸方向の応力$\sigma_2$が生じている。
\begin{eqnarray}
2\pi r t \sigma_2 &=& \pi r^2 P \\
\sigma_2 &=& \frac{rP}{2t}=\frac{dP}{4t}
\end{eqnarray}
一方、円筒部分の上半分のみを切り出してみると円筒胴の内壁に圧力が作用することにより、
円周応力$\sigma_1$が生じている。
これから、内圧による垂直方向の分力の合計が、
円周応力$sigma_1$により壁に発生した力と釣り合うことから、次式により計算できる。
\begin{eqnarray}
2tb\sigma_1 &=& 2Prb \\
\sigma_1 &=& \frac{rP}{t}= \frac{dP}{2t}
\end{eqnarray}
以上より薄肉円筒の設計をする場合、$\sigma_1=2\sigma_2$なので、
円周応力のみを計算すれば許容応力範囲内に収まる。
円筒胴の計算式は以下のようになる。
\begin{eqnarray}
t = \frac{PD_i}{2\sigma_a \eta - 1.2P}
\end{eqnarray}
容器の材質はステンレス鋼(JIS G 3214 : SUS304)を使用し、引っ張り応力は$520[N/mm^2]$とした。
実証機の設計諸元、寸法及び概略図を表と図に示す。



