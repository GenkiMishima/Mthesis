\section{実証機モデルの性能計算}
実験によって得られた計測データを元に実証機モデルの性能計算を評価し、
実証器モデルの寸法を決定した。
性能計算は燃焼室圧力と酸化剤質量流量を仮定し以下の要領で計算を行った。
\begin{itemize}
\item 酸化剤質量流束より燃料流量を推定\\
酸化剤質量流束は燃焼室に噴射する酸化剤質量流量と燃料ポート断面積から求める。
燃料流量は実験によって得られた平均燃料後退速度と酸化剤質量流量より計算する。
\item 燃焼室圧力、O/FよりnasaCEAで化学平衡計算\\
燃料及び酸化剤流量よりO/Fを決定し、nasaCEAを用いて化学平衡計算を行うことで、燃焼室内の平衡状態を求めた。
\item 計算結果より平均燃料グレイン形状を推定\\
平均燃料後退速度より燃料グレインポート直径の平均値を求めた。
\item 燃焼終了時間まで繰り返す
\end{itemize}
この計算を各タイムステップ毎に行うことで、実証器モデルの性能計算を行った。
検討しているハイブリッドロケットエンジンの酸化剤質量流量は最大$10[kg/s]$を想定しているため(\ref{tab:C4LOXbase})、
本検討でも酸化剤質量流量は$10[kg/s]$とした。
燃焼室圧力は$5[MPaA]$と仮定した。
燃焼時間$15[s]$とした本計算の計算結果を図\ref{C4Result}に示す。
反応生成物は$H_2O,CO_2,O_2$が大部分を占めており、本反応における平衡状態では$H_2O,CO_2,O_2$が支配的であることが確認できた。
O/Fを50程度に保つことができ、ノズルから排出される酸素温度も$380~390[K]$になっていることから
反応生成物全てを気体状態で排出できることが確認できた。

実証機の設計諸元と寸法及び概略図を表\ref{tab:RealModel}と図\ref{fig:LargeModel}に示す。
