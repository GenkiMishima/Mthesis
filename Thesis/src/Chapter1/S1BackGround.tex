\subsection{研究背景}
近年宇宙産業は新興国も含めて世界中で需要が拡大しており、
アメリカの調査会社は超小型衛星(1-50kg)需要が2020年までに400−500機ほどになると予想している。\cite{nano/micro}
我が国においても新たな産業需要を見込んでおり、
従来の通信・放送に加え、災害監視・環境観測・農林漁業・国土監視・資源探査・安全保障などの地球観測の分野での利用拡大が予想されている。
その一方で宇宙輸送技術は発展途上段階にあり非常に高価なものである。
現在打ち上げられている超小型衛星の90\%は大型衛星に相乗りして打ち上げられたものであり、
打ち上げ機会また条件が限られているため、
この超小型衛星の需要を満たすことができないと予想されている。
