\section{本研究の目的}
本研究は到達高度100kmのSOFTハイブリッドロケットに搭載する
プリバーナ方式の液体酸素気化装置が研究対象である。
検討するハイブリッドロケットの概念設計の基本データを図\ref{fig:Baseline}に示す。
検討する気化装置は酸化剤をLOX,固体燃料をアクリル樹脂(PMMA)とし、
気化器内で高O/F環境下でのLOX/PMMA境界層燃焼により燃焼ガスを発生させ、
LOXと燃焼ガスを混合させることで気化を行うものである。
本気化器では混合促進のためにバッフルプレートを用いている。
本研究で搭載を検討しているハイブリッドロケットの酸化剤はLOXであるため、
本気化器においても酸化剤はLOXとし、
ハイブリッドロケットの固体燃料として多くの実績があり、
High-Speedカメラによる可視化が行えるPMMAを固体燃料として使用した。
この気化器に関して、小流量ケースでの燃焼流の可視化及び基礎データ取得実験を行い、
実験結果を元に実証機レベルの設計計算を行うことで実現性について調査することが目的である。
