\section{研究背景}
%\subsection{研究背景}
近年宇宙産業は新興国も含めて世界中で需要が拡大しており、
アメリカの調査会社は超小型衛星(1-50kg)需要が2020年までに400−500機ほどになると予想している。\cite{nano/micro}
我が国においても新たな産業需要を見込んでおり、
従来の通信・放送に加え、災害監視・環境観測・農林漁業・国土監視・資源探査・安全保障などの地球観測の分野での利用拡大が予想されている。
その一方で宇宙輸送技術は発展途上段階にあり非常に高価なものである。
現在打ち上げられている超小型衛星の90\%は大型衛星に相乗りして打ち上げられたものであり、
打ち上げ機会また条件が限られているため、
この超小型衛星の需要を満たすことができないと予想されている。

%\subsection{ハイブリッドロケット}
ハイブリッドロケットは科学推進ロケットの一つである。
科学推進ロケットは推進剤の酸化剤と燃料を燃焼させることで生じる高温・高圧の燃焼ガスをノズルにより膨張させることで、高速噴流を作り、推力を得るロケットである。
燃焼形態は予混合燃焼と拡散燃焼に分類することができ、


近年宇宙産業は新興国も含めて世界中で需要が拡大しており、
アメリカの調査会社は超小型衛星 (1-50kg) 需要が2020年までに400-500機ほどになると予想している。\cite{nano/micro}
我が国においても新たな産業需要を見込んでおり、
従来の通信・放送に加え、災害監視・環境観測・農林漁業・国土監視・資源探査・安全保障などの地球観測の分野での利用拡大が予想されている。
その一方で宇宙輸送技術は発展途上段階にあり非常に高価なものである。
現在打ち上げられている超小型衛星の90\%は大型衛星に相乗りして打ち上げられたものであり、
打ち上げコストが高くまた機会・条件が限られているため、
超小型衛星の需要を満たす宇宙輸送機を供給することができないと予想されている。
この供給量を増やすためには以下のコスト削減手段が考えられる。
\begin{itemize}
	\item 生産性と効率の改善
	\begin{itemize}
		\item 各構成要素の規格化と一般化
		\item 量産システムの構築
	\end{itemize}
	\item 商業化・自由競争による促進
	\item 技術的なイノベーション
	\begin{itemize}
		\item 再使用ロケット
		\item 3-Dプリンターによる製造
		\item ロケットの潜在的危険性を取り除いた(Delethalized)設計 
	\end{itemize}
\end{itemize}
過去25年間 (1980年-2004年) における世界の宇宙機打ち上げの調査により、
打ち上げ失敗は推進システムによるものが優位を占めていることがわかっている。(図\ref{fig:Success/Failure})\cite{failure}
\\
科学推進系ロケットの燃焼メカニズムは容易に火炎領域を拡大することのできる予混合燃焼と
火炎領域の拡大が難しい拡散燃焼の大きく2種類に分けることができる。
固体燃料ロケットは燃料グレインと酸化剤を予め混合した推進薬を用いて燃焼させる予混合燃焼を起こす。
液体燃料ロケットは燃料と酸化剤をインジェクター付近で緊密に混ぜ合わせた推進薬を使用する。
微視的には拡散燃焼と予混合燃焼が同時に存在しているが、
巨視的には混合が急速に行われるため、燃焼過程は予混合燃焼のように振る舞う。
一方ハイブリッドロケットは燃料グレイン表面に生成される境界層内で燃料と酸化剤の燃焼が行われ、
この燃焼過程は拡散範囲が限らてれる境界層燃焼となる。
この違いはロケットの打ち上げ失敗の際に顕著に現れ、
予混合燃焼は爆発するのに対して、拡散燃焼は爆発しない。(図\ref{fig:DiffExplo})
\\
この予混合燃焼を用いるロケットの安全性を確保するためには
有資格者による監視・新しい施設や設備の導入・申請手続きなどの
危険物に対する対策が必要となり、生産コストが増加するが、
拡散燃焼を用いるDelethalizedしたロケットではこれらのコストを削減することが期待されている。
\\
安定・継続的な境界層燃焼を維持するためには、境界層中の火炎から安定・継続的に熱が供給されて燃料の気化が起き、
酸化剤蒸気と拡散混合して火炎が維持されなければならない。
これは人為的に活性状態を維持する必要があり、この燃焼過程には高い安全性が確保される等の長所あるが、
固体燃料の後退速度の低さ・C*の低さ・燃料流量を特設制御できないなどの大きな短所が生まれる。
境界層燃焼の概念図は図\ref{fig:BLC}に示す。
\\
ハイブリッドロケットは以下に示す特徴を有している。
\begin{itemize}
	\item 境界層燃焼機構を有するため高い安全性がある。
	\item 燃料はプラスチック・ロウ・ゴム、酸化剤がLOX,N2Oなどであるため、一般工業レベルのシンプルな品質管理となる。
	\item 高い安全性とシンプルな品質管理なので経済性が高くなる。
	\item 炭化水素と酸素の反応は環境に優しく、また爆発しないためスペースデブリになりにくい。
	\item 酸化剤流量を調節することで、推力調整が可能となる。
\end{itemize}


