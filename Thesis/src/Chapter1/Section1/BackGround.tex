\subsection{研究背景}
現在、様々な国が国家プロジェクトとして宇宙開発を行っており、各国の開発する大型で高性能な宇宙機及び衛星が注目を浴びている。
大型衛星は様々な機器を搭載することで、高度で、多岐にわたる観測や実験んが可能である。
その一方で、開発コストが高いため、開発に慎重にならざるをえず、信頼性に低い新技術が採用されにくいことや開発期間が長いため、科学観測等で球を等するミッションに対応できないといった欠点を有している。
これらの理由から、機能を絞った小型衛星の開発が盛んに行われるようになった。
小型衛星は機能を絞ることで、短期間での開発が可能となり、開発コストも低くすることができる。

