\subsection{LOX/PMMA反応機構}
本実験で用いるアクリル樹脂(PMMA:$(C_5H_8O_2)$)の写真を図\ref{fig:PMMA}に示す。
PMMAはハイブリッドロケットの固体燃料として多くの実績があり、
透過率が非常に高いため供試体内部の可視化実験が行える。
グレイン形状は外形50mm、内径30mmの円筒形になっており長さの違う3つの形状を製作した。
表\ref{tab:PMMA}に示す。
PMMAの活性化エネルギーは$-442.14[kj/mol]$である。
\\
LOX/PMMA反応式は以下のようになる。
\begin{equation}
6O_2+C_5H_8O_2 -> 4H_2O+5CO_2-442.14[kJ/mol]
\end{equation}

液体酸素(LOX)の潜熱と温度を飽和蒸気圧の関数として図\ref{fig:LOXBoil}に示す。
気化された酸素温度を400-500[K]にするために、
酸化剤-燃料質量比率(O/F)=50となるように設計した。 


