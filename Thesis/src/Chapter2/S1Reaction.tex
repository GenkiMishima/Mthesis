\subsection{LOX/PMMA反応機構}
本実験で用いるアクリル樹脂(PMMA:$[CH_2C(CH_3)(COOH_3)]_n$)の写真を図\ref{fig:PMMA}に示す。
PMMAはハイブリッドロケットの固体燃料として多くの実績があり、
透過率が非常に高いため供試体内部の可視化実験が行える。
グレイン形状は外形50mm、内径30mmの円筒形になっており長さの違う3つの形状を製作した。
表\ref{tab:PMMA}に示す。
PMMAの活性化エネルギーは$-442.14[kj/mol]$である。

\begin{table}[htb]
\begin{center}
\caption{PMMA形状}
\begin{tabular}{|l|c|c|c|c|} \hline
 & 外径 & 内径  & 長さ  & 内部の表面積 \\ \hline
 & [mm] & [mm]  & [mm]  & [mm\^2]      \\ \hline
1&      &       & 30    &706.5         \\ \cline{4-5}
2& 50   & 30    & 15    &1413          \\ \cline{4-5}
3&      &       & 7.5   &2826          \\ \hline
\end{tabular}
\label{tab:PMMA}
\end{center}
\end{table}

\\
\  
液体酸素(LOX)の潜熱と温度を飽和蒸気圧の関数として図\ref{fig:LOXBoil}に示す。

\\
\  
LOX/PMMA反応は酸化剤-燃料比率(O/F)=50となるように設計した。 

